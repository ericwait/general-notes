\documentclass[11pt,spacing=single]{article}

%%%%%%%%%%%%%%%%%%%%%%%%%%%%%%%%%%%%%%%%%%%%%%%%%%%%%%%%%%%%%%%
% Packages
%%%%%%%%%%%%%%%%%%%%%%%%%%%%%%%%%%%%%%%%%%%%%%%%%%%%%%%%%%%%%%%
\usepackage{geometry}
\usepackage{graphicx}
\usepackage{pdflscape}
\usepackage{titlepic}
\usepackage{amsthm,amsmath,amssymb}
\usepackage{sansmathfonts}
\usepackage{listings}
\usepackage[shortlabels]{enumitem}
\usepackage{hyperref}
\usepackage{nameref}
\usepackage{wrapfig}
\usepackage{lipsum}
\usepackage{cite}
\usepackage[english]{babel} % English blindtext
\usepackage{blindtext} % macros for producing dummy output
\bibliographystyle{ieeetr}
\usepackage{fancyvrb} % nicer verbatim handling
\usepackage{sansmathfonts}
\renewcommand{\familydefault}{\sfdefault}

\usepackage[colorinlistoftodos,textwidth=20mm,shadow]{todonotes}
% \usepackage[disable]{todonotes}
% \usepackage{draftwatermark}
% \SetWatermarkText{\parbox{\textwidth}{\centering DRAFT v0.1}}
% \SetWatermarkScale{1.0}

%%%%%%%%%%%%%%%%%%%%%%%%%%%%%%%%%%%%%%%%%%%%%%%%%%%%%%%%%%%%%%%
% Local defines
%%%%%%%%%%%%%%%%%%%%%%%%%%%%%%%%%%%%%%%%%%%%%%%%%%%%%%%%%%%%%%%
\DeclareMathOperator{\sgn}{sgn}
\newcommand{\ith}{\ensuremath{i^\text{th}} }
\newcommand{\jth}{\ensuremath{j^\text{th}} }
\newcommand{\nth}{\ensuremath{n^\text{th}} }
\newcommand{\blditm}[1]{\item{\textbf{#1}}}

% Use \qst command for questions that need answering
\newcommand{\qst}[1]{\todo[color=magenta]{#1}}
% Use \lookup for items that need further research
\newcommand{\lookup}[1]{\todo[color=green]{#1}}
% Use \tsk for tasks that still need to be completed
\newcommand{\tsk}[1]{\todo[inline,color=cyan]{#1}}
% Use \idea for ideas not ready to be in text
\newcommand{\idea}[1]{\todo[inline,color=yellow]{#1}}
% Use \err to highlight were there are serious problems
\newcommand{\err}[1]{\todo[inline,color=red]{#1}}
% use to put a note in the margin
\newcommand{\sidenote}[2]{\todo[color=#1,size=\tiny]{#2}}

\newcommand{\etal}{\textit{et al. }}
\newcommand{\ie}[1]{\textit{i.e.} {#1}}
\newcommand{\eg}[1]{\textit{e.g.} {#1}}

% Use ``\cite{NEEDED}'' to get Wikipedia-style ``citation needed'' in document
\usepackage{ifthen}
\let\oldcite=\cite
\renewcommand\cite[1]{\ifthenelse{\equal{#1}{NEEDED}}{\textsuperscript{\texttt{\color{red}[citation~needed]}}\todo{citation needed}}{\oldcite{#1}}}

%%%%%%%%%%%%%%%%%%%%%%%%%%%%%%%%%%%%%%%%%%%%%%%%%%%%%%%%%%%%%%%
% Preamble
%%%%%%%%%%%%%%%%%%%%%%%%%%%%%%%%%%%%%%%%%%%%%%%%%%%%%%%%%%%%%%%

\author{Eric Wait}
\title{General Notes}
\date{\today}

\begin{document}
	\maketitle
	\newpage

	\listoftodos
	\newpage

	\section{Windows}
		\subsection*{Links}
			How to stop windows 10 from rebooting automatically \href{https://www.windowscentral.com/how-prevent-windows-10-rebooting-after-installing-updates}{https://www.windowscentral.com/how-prevent-windows-10-rebooting-after-installing-updates}

	\section{Linux}
		\subsection*{Git}
			Download current version \href{https://mirrors.edge.kernel.org/pub/software/scm/git/}{https://mirrors.edge.kernel.org/pub/software/scm/git/}
			If you need to install a local version that is more up-to-date than what is available system wide:
			\begin{enumerate}
				\item Uncompress the git package
				\item Within the new git dir, run \texttt{make} then \texttt{make install}
				\item Put \texttt{\~/bin} on your path within \texttt{.bashrc}
			\end{enumerate}
			Download current version of gitlfs \href{https://git-lfs.github.com/}{https://git-lfs.github.com/}
			\begin{enumerate}
				\item Uncompress the git-lfs package
				\item Within the new git-lfs dir, edit the \texttt{install.sh} file so that \texttt{prefix} is pointing to somewhere in your user directory.
				\item Run the install script
				\item Place a link within the your user bin directory, \eg{\texttt{ln -s /home/user/git-lfs/local/bin/git-lfs /home/user/bin/git-lfs}}
				\item Put \texttt{\~/bin} on your path within \texttt{.bashrc}, if not already
			\end{enumerate}
			Download a gui like gitkraken \href{https://www.gitkraken.com/download}{https://www.gitkraken.com/download}
			Place a link within your user bin directory, \eg{\texttt{ln -s /home/user/gitkraken/gitkraken /home/user/bin/gitkraken}}
			\textbf{\color{red} If installed in your local user space, you will be responsible to update packages!}

	\section{Git}
		\subsection*{Scripts for Custom Actions}
			\subsubsection*{Windows}
				Gitlfs fetch all, make \texttt{gitlfsFetchAll.bat} with:\\
				\texttt{FOR /F "delims==" \%\%a IN ('git remote') DO (git lfs fetch --all \%\%a)}\\

				\noindent
				Gitlfs push all, make \texttt{gitlfsPushAll.bat} with:\\
				\texttt{FOR /F "delims==" \%\%a IN ('git remote') DO (git lfs push --all \%\%a)}\\

				\noindent
				Git push all, make \texttt{gitPushAll.bat} with:\\
				\texttt{FOR /F "delims==" \%\%a IN ('git remote') DO (git push \%\%a --all)}

			\subsubsection*{Linux/Mac}
				Gitlfs fetch all, make \texttt{gitlfsFetchAll.sh} with:\\
				\texttt{\#!/bin/bash\\ git remote \textbar xargs -n1 git lfs fetch --all}\\

				\noindent
				Gitlfs push all, make \texttt{gitlfsPushAll.sh} with:\\
				\texttt{\#!/bin/bash\\ git remote \textbar xargs -n1 git lfs push --all}\\

				\noindent
				Git push all, make \texttt{gitPushAll.sh} with:\\
				\texttt{\#!/bin/bash\\ git remote \textbar xargs -n1 git push --all}

	\newpage
	\bibliography{ref.bib}
\end{document}
